\section{Theory}
    Every light source has its own emission spectrum. (A spectrum of the light frequencies that the object emits.)
\newline
    Each of these frequencies has a wave length. When these "collection of waves" called a wave front meets a slit with a width smaller than the wave length, it diffracts.
\newline
    The excact angle of diffraction depends on the frequency of the light wave.
\newline
    By measuring the angle difference, we can obtain knowledge of the exact frequency that was emitted.
\newline
\newline
    When light has passed through a slit of a width comparable to or smaller than the wave length, it diffracts completely, causing the slit to appear as a point light source emitting spherical light waves.
\newline
\newline
    Interfering light waves (waves that meet each other) will interfere with each other. This causes constructive or deconstructive interference depending on whether or not the waves are in phase at the moment of interference. 
\newline    
    Since the phase of the wave changes as the waves travel through space when 2 waves have traveled an equal amount of distance (and they are of the same frequency), they will be in phase.
\newline
    At the angle of diffraction, all waves will therefore approximately be in phase and cause constructive interference.
\newline
    Slight deviations of this angle will cause some waves to have traveled a longer distance, hence being out of phase, causing a dimmer image.
\newline
    A formalization of the dimming of the image is called the point-spread function, where a point, like a light source, seems to have been blurred, increasing its width and height.
\newline
    The same concept explains why 2 objects of smaller size than the point spread function, who are situated close to each other seem to overlap on an image, since their point spread functions overlap.
\newline
\newline
    We will use the Rayleigh criterion to determine the resolution of our image.


    

    
    


