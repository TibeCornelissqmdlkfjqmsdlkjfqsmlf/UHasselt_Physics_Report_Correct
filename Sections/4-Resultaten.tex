\clearpage
\section{Results}
    \subsection{Transfer function}
        \subsubsection{For an infinitely wide grating with infinitely narrow \texorpdfstring{$\delta$}{Lg} slits.}
            
            \begin{equation}
                \mathcal{F}[f(x)] = \sum_{n=-\infty}^{\infty} \delta(\omega - 2n\frac{\pi}{d})
            \end{equation}

        \subsubsection{For a finite wide grating with infinitely narrow \texorpdfstring{$\delta$}{Lg} slits.}
            
            \begin{equation}
                \mathcal{F}[f(x)] = \mathcal{F}[Rect(\frac{x}{b})\cdot
                \sum_{n=-\infty}^{\infty} \delta(x-nd)]
            \end{equation}
            
            \begin{equation}
                \mathcal{F}[f(x)] = b \cdot \sinc{ \frac{\omega \cdot b}{2}} * \sum_{n=-\infty}^{\infty} \delta(\omega - 2n\frac{\pi}{d}) 
            \end{equation}
            
        \subsubsection{For a finitely wide grating with a slit of width b}
        
            \begin{equation}
                \mathcal{F}[f(x)] = b \cdot \sinc\left(\frac{\omega b}{2}\right)   \cdot [ \sinc\left(\frac{\omega W}{2}\right) * \sum_{n=-\infty}^{\infty} \delta(\omega - 2n\frac{\pi}{d}) ] 
            \end{equation}
            
   \subsection{Amount of Slits per mm}

        The width of each slit is denoted by \( d \):
        
        \begin{equation}
            d = 1.67 \cdot 10^{-6} \, m
            \label{eq:WidthSlit}
        \end{equation}
        
        The approximate number of slits per millimeter is:

        \begin{equation}
            N \approx 599
            \label{eq:SlitCount}
        \end{equation}
        
    \subsection{Wave length emission lines}
        \subsubsection{Measured angles}
            \begin{table}[H]
                \centering
                \renewcommand{\arraystretch}{1.3}
                \begin{tabular}{l|c}
                    \textbf{Color} & \textbf{Measured Angle (°)} \\
                    \hline
                    Violet & $14.0^\circ$ \\
                    Blue & $15.0^\circ$ \\
                    Green & $17.0^\circ$ \\
                    Yellow 1 & $19.0^\circ$ \\
                    Yellow 2 & $20.2^\circ$ \\
                \end{tabular}
                \caption{Measured angles for different emission lines (First order).}
            \end{table}

            \begin{table}[H]
                \centering
                \begin{tabular}{l|c}
                     \textbf{Color} & \textbf{Measured Angle (°)}\\
                     \hline
                     Violet & $29^\circ$\\
                     Blue & $31^\circ$\\
                     Green & $36^\circ$\\
                     Yellow 1& $41^\circ$\\
                     Yellow 2& $43^\circ$\\
                \end{tabular}
                \caption{Measured angles for different emission lines (Second order).}
                \label{tab:measured_angles_second}
            \end{table}
            
        \subsubsection{Theoretical derivation wave lengths}
        
            Using the formula $\sin{\theta} = \frac{n\lambda}{d}$, under the assumption that the wavelength of violet is known.

            \begin{equation}
                \lambda_{violet} = 404.7\mu m
            \end{equation}

            %Table theoretical derived wave lengths First order
            \begin{table}[H]
                \centering
                \begin{tabular}{c|c}
                     \textbf{Color} & \textbf{Wave length ($nm$)} \\
                     \hline
                     Violet &  404.7\\
                     Blue &  432.2\\
                     Green & 488.3\\
                     Yellow 1 & 543.7\\
                     Yellow 2 & 576.6\\
                \end{tabular}
                \caption{Wave lengths first order}
                \label{tab:wavelengths}
            \end{table}

            %Table theoretical derived wave lengths Second order
            \begin{table}[H]
                \centering
                \begin{tabular}{c|c}
                    \textbf{Color} & \textbf{Wave length ($nm$)} \\
                    \hline
                    Violet &  404.7\\
                    Blue &  432.2\\
                    Green & 488.3\\
                    Yellow 1 & 543.7\\
                    Yellow 2 & 576.6\\
                \end{tabular}
                \caption{Wave lengths second order}
            \end{table}
    \subsection{Resolution}
        Using the Rayleigh criterion, the resolving power is given by.
        
        \begin{equation}
            \mathcal{R} = \frac{\lambda}{\Delta \lambda}
        \end{equation}
        \begin{equation}
            W = \frac{\mathcal{R} \cdot d }{m} = 0.457mm
        \end{equation}
        
        Which is the minimum width of the grating to be able to distinguish 2 waves of length $\lambda_1 = 576.96nm$ and $\lambda_2 = 579.07nm$.