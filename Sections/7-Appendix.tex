

\section*{Appendix: Fourier Transform of the Rectangular Window Function}

\begin{equation}
\text{rect}(a) = 
\begin{cases}
1 & \text{if } |a| \leq \frac{1}{2}, \\
0 & \text{otherwise.}
\end{cases}
\end{equation}

\begin{equation}
F(\omega) = \int_{-\infty}^{\infty} f(t)\, e^{-i \omega t} \, dt.
\end{equation}

Since \( f(t) = 0 \) outside the interval \(\bigl[-\tfrac{\tau}{2}, \tfrac{\tau}{2}\bigr]\), the integral reduces to:
\begin{equation}
F(\omega) 
= \int_{-\frac{\tau}{2}}^{\frac{\tau}{2}} e^{-i \omega t} \, dt.
\end{equation}

Evaluating this integral gives:
\begin{equation}
F(\omega) 
= \left[ \frac{e^{-i \omega t}}{-i \omega} \right]_{-\frac{\tau}{2}}^{\frac{\tau}{2}}
= \frac{1}{-\,i\,\omega} 
\left( e^{-\,i\,\omega \,\frac{\tau}{2}} \;-\; e^{i\,\omega \,\frac{\tau}{2}} \right).
\end{equation}

Using 
\(\sin(\alpha) = \tfrac{1}{2i}\bigl(e^{i\alpha} - e^{-\,i\alpha}\bigr)\), it follows that:
\begin{equation}
F(\omega) 
= \frac{2}{\omega} \,\sin\!\Bigl(\tfrac{\omega \,\tau}{2}\Bigr).
\end{equation}

Expressing this in terms of the \(\mathrm{sinc}\) function yields:
\begin{equation}
F(\omega) 
= \tau \,\mathrm{sinc}\Bigl(\tfrac{\omega\,\tau}{2\pi}\Bigr),
\quad \text{where } 
\mathrm{sinc}(x) = \frac{\sin(\pi x)}{\pi x}.
\end{equation}

\section*{Appendix: Fourier Transform of the Dirac Comb}

Consider the function (sometimes referred to as the "comb function"):
\begin{equation}
f(x) = \sum_{n=-\infty}^{\infty} \delta\bigl(x - n\,d\bigr),
\end{equation}
where \(d\) is the distance between consecutive \(\delta\)-pulses.

The Fourier transform \(F(\omega)\) of a function \(f(x)\) is defined by:
\begin{equation}
F(\omega) 
= \int_{-\infty}^{\infty} f(x)\, e^{-\,i\,\omega\,x}\,dx.
\end{equation}

Substitute \(f(x)\) into the definition:
\begin{equation}
F(\omega)
= \int_{-\infty}^{\infty} 
\left(\sum_{n=-\infty}^{\infty} \delta\bigl(x - n\,d\bigr)\right) 
e^{-\,i\,\omega\,x} \, dx.
\end{equation}
Under suitable conditions, the sum and the integral may be interchanged:
\begin{equation}
F(\omega)
= \sum_{n=-\infty}^{\infty} 
\int_{-\infty}^{\infty} \delta\bigl(x - n\,d\bigr)\, e^{-\,i\,\omega\,x}\, dx.
\end{equation}

The sifting property of the Dirac delta implies:
\[
\int_{-\infty}^{\infty} \delta\bigl(x - a\bigr)\, g(x)\,dx 
= g(a).
\]
By setting \( a = n\,d \):
\begin{equation}
F(\omega)
= \sum_{n=-\infty}^{\infty} e^{-\,i\,\omega\,n\,d}.
\end{equation}

The expression
\(\sum_{n=-\infty}^{\infty} e^{-\,i\,\omega\,n\,d}\)
is an infinite exponential series and can be rewritten as follows:
\begin{equation}
\sum_{n=-\infty}^{\infty} e^{-\,i\,\omega\,n\,d}
= \frac{2\pi}{d} 
\sum_{k=-\infty}^{\infty} 
\delta\!\Bigl(\omega - \tfrac{2\pi k}{d}\Bigr).
\end{equation}

Hence, the Fourier transform of the Dirac comb is:
\begin{equation}
F(\omega)
= 
\frac{2\pi}{d}
\sum_{k=-\infty}^{\infty} 
\delta\!\Bigl(\omega - \tfrac{2\pi k}{d}\Bigr).
\end{equation}

In summary,
\[
\mathcal{F}\Bigl\{\sum_{n=-\infty}^{\infty} \delta(x - n\,d)\Bigr\}
= 
\frac{2\pi}{d} \sum_{k=-\infty}^{\infty} \delta\!\Bigl(\omega - \tfrac{2\pi k}{d}\Bigr).
\]
